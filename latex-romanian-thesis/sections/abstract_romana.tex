\thispagestyle{plain}
\setlength{\parskip}{1.5em}

\vspace*{\fill}

\begin{center}
	\Large \textbf{Abstract}	
\end{center}

Recunoașterea facială este o problemă foarte cunoscută care a fost cercetată in mod special în ultima jumătate de deceniu. Modalitățile actuale de abordare a acestei probleme se bazează pe paradigma de învățare \textit{deep}. Ele antreanează de la un capăt la altul rețele neuronale convoluționale (CNN) cu milioane de example astfel încât rețeaua antrenată sa discrimineze între perechi de fețe identice si diferite.

În proiectul pe care prezenta lucrare de licență își propune sa îl descrie, am folosit o asemenea arhitectura de tip \textit{deep} pentru a crea un sistem de autentificare facială. Având în vedere aplicațiile în securitate ale acestuia, am adăugat o componentă de validare a fețelor bazată pe modele locale binare (LBP) pentru a combate atacuri precum poze printate sau videoclipuri redate pe dispozitive mobile

Am reușit să realizăm un sistem de recunoaștere faciala de încredere care se pretează pentru utilizarea în producție. În aces sens, am utilizat implementarea public disponibila Openface \cite{amos2016openface} a arhitecturii de tip CNN FaceNet \cite{SchroffKP15} care a atins precizia de \textbf{99.63\%} pe setul de date Labeled Faces in the Wild. 
În ce priveste validarea facială, am implementat construirea pe baza modelelor binare locale (LBP) uniforme a vectorului de caracteristici. Acesta a fost utilizat ca date de intrare pentru o mașină cu vector suport pentru a obține clasificarea ca față reală sau atac. Implementarea noastră a dat rezultate bune pe seturi de date din aceeași bază de date. Deoarece datele public disponibile nu au o putere de generalizare corespunzătoare, această parte a sistemului nu poate încă fi considerată utilizabilă în producție.

\vspace*{\fill}

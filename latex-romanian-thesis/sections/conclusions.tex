\chapter{Conclusions}
In this thesis we studied the use of local binary patterns in combating face spoof attacks of a facial authentication system along with the ability of neural networks to learn a face mapping function that would give reliable results in matching two or more faces. 

In terms of face spoof validation, we observed that LBPs can be successfully used for this purpose when the training data is collected in the same conditions as the test data. This can be seen in the section \ref{section:face_spoof_databases} where we did experiments on a number of databases and we had satisfactory results. On the other hand, our implementation and training data lacked generalization power as it resulted from out cross-database experiments \ref{section:cross_database_testing}. In this case, the results were disappointing: only one in four test folds gave decent results, where the other three were a little better than a random choice. I believe the detection of spoof faces is, in the moment, a really difficult problem given that not even a human can easily differentiate between a photo of a live face and spoof one. Having in mind the great advances in the deep learning field at the moment, a new possible approach of the problem could be collecting sufficient data in the wild and train an CNN architecture with the purpose of learning a classifier that would differentiate the two classes. The major challenge here would be the collection of attack images given that this case is not common in real life.

Regarding the face recognition part, we can state that the great results obtained in our days in terms of face identification are due to use of deep learning architectures. The key elements that enables such systems to reach state of the art performance are the large training datasets (as stated in \cite{LiuDBH15} the size of the training data for such networks ranges from 2M to 15M in the academic field to about 200M in the industry). This is made possible by the fact that in our days, collecting data at such a scale is not an impossible challenge. Even though the results of deep learning methods are \textit{only} comparable to the humans results, meaning that, for example, a pair of twins that a human cannot differentiate, neither a neural network can, I am confident that such systems can be deployed in production and be reliable. I also suggest that these systems should allow for continuous data gathering given that it is the solid way of advancement in the current approach.

